% !TeX root = ./05_answers.tex

\documentclass[english]{scrartcl}
\usepackage[T1]{fontenc}
\usepackage[utf8]{inputenc}
\usepackage{multicol}
\usepackage{paralist}
\usepackage{hyperref}
\usepackage{listings}
\usepackage{fvextra} % loads also fancyvrb
\usepackage{pgfgantt}
\usepackage{amsmath}
\usepackage{xcolor}
\usepackage{dirtytalk}
\usepackage{xparse}
\usepackage{xpatch}
\usepackage[bottom=1.2in,top=1.2in]{geometry}
\usepackage{tikz}
\usetikzlibrary{arrows,automata,positioning}
\hypersetup{
    colorlinks,
    citecolor=black,
    filecolor=black,
    linkcolor=black,
    urlcolor=black
}
\lstset{
  basicstyle=\ttfamily,
  mathescape
}

\DeclareMathVersion{ttmath}
\DeclareSymbolFont{latinletters}{OT1}{\ttdefault}{m}{n}
%\SetSymbolFont{latinletters}{ttmath}{OT1}{\ttdefault}{m}{n}
\SetSymbolFont{letters}{ttmath}{OML}{ccm}{m}{it}
\SetSymbolFont{symbols}{ttmath}{OMS}{ccsy}{m}{n}
\SetSymbolFont{largesymbols}{ttmath}{OMX}{ccex}{m}{n}

% https://tex.stackexchange.com/questions/150965/insert-symbols-inside-verbatim-mode-latex/156419

\newcommand{\changeletters}{%
  \count255=`A
  \advance\count255 -1
  \loop\ifnum\count255<`Z
    \advance\count255 1
    \mathcode\count255=\numexpr\number\symlatinletters*256+\count255\relax
  \repeat
  \count255=`a
  \advance\count255 -1
  \loop\ifnum\count255<`z
    \advance\count255 1
    \mathcode\count255=\numexpr\number\symlatinletters*256+\count255\relax
  \repeat
  \count255=`0
  \advance\count255 -1
  \loop\ifnum\count255<`9
    \advance\count255 1
    \mathcode\count255=\numexpr\number\symlatinletters*256+\count255\relax
  \repeat
}

\xapptocmd{\ttfamily}{\mathversion{ttmath}\changeletters}{}{}

\setkomafont{disposition}{\normalfont\bfseries}
\definecolor{light-gray}{gray}{0.95}

\begin{document}

\NewDocumentCommand{\codeword}{v}{%
\colorbox{light-gray}{\texttt{\textcolor{black}{#1}}}%
}

% == START: TITLE == %

\subtitle{6CCS3CFL - Compilers \& Formal Languages}
\title{Homework 5}
\author{Finley Warman}
\date{\today}

\maketitle

% == END: TITLE == %
% == START: CONTENTS == %

\tableofcontents
\par\noindent\rule{\textwidth}{0.4pt}

% == END: CONTENTS == %

\newpage

% MAIN CONTENT:

\addcontentsline{toc}{section}{Question 1}
\section*{Question 1}
Q: How would you show a property $P(r)$ holds for all regular expressions $r$, by structural induction?
\\
\\
A:
\begin{itemize}
    \item As a basis, show $P(r)$ holds for all terminal cases, i.e. \verb~ rBase ::= 0 | 1 | c ~.
    \item Let r1 and r2 be regexes, and suppose P(r1) and P(r2) hold [This is the inductive hypothesis, IH].
    \item Induction over the remaining cases (which are inductively defined) to show property holds on by applying IH: \\
        \verb~ rInd ::== r1+r2 | r1.r2 | r* ~.
    \item Thus all cases are covered, which is sufficient to prove a property by structural induction.
\end{itemize}

\addcontentsline{toc}{section}{Question 2}
\section*{Question 2}
Q: Define a regular expression, $ALL$, that can match every string. \\
Write this in terms of \verb~ r ::= 0 | 1 | c | r1+r2 | r1.r2 | r* | ~r ~
\\
A: \verb| ALL = ~(0)|

\addcontentsline{toc}{section}{Question 3}
\section*{Question 3}
Q: Define the expressions $r^+$, $r^?$, $r^{\{n\}}$, $r^{\{m,n\}}$ in terms of basic regular expressions. \\

A: \begin{itemize}
    \item $r^+$ = r . r*
    \item $r^?$ = \textbf{1} + r
    \item $r^{\{n\}}$ = $r_1$ . $r_2$ . \dotso . $r_n$
    \item $r^{\{m,n\}}$ = ($r_1$ . $r_2$ . \dotso . $r_m$) . ( 1 + ($r_{m+1}$) + ($r_{m+1}$ . $r_{m+2}$) + ($r_{m+1}$ . \dotso . $r_{n}$) )
    \item (where $r_i = r$ for all $i$)
\end{itemize}

\addcontentsline{toc}{section}{Question 4}
\section*{Question 4}
Q: Given the regular expressions for lexing a language with identifiers, parenths, numbers (postive/negative), and operators \verb~+, -, *~: \\
\\
\newpage
A:
\begin{verbatim}
  DIGIT       = RANGE("0123456789")
  START_DIGIT = RANGE("123456789")
  NUMBER      = OPT(CHAR('-')) . (DIGIT + (START_DIGIT . DIGIT*))
                    [Numbers are implicitly positive, or explicitly negative]

  LPAREN      = CHAR('(')
  RPAREN      = CHAR(')')

  OPERATOR    = CHAR('+') + CHAR('-') + CHAR('*')

  LOWERCASE   = RANGE("abcdefghijklmnopqrstuvwxyz")
  ID          = LOWERCASE . LOWERCASE*

  LANG        = ("num":NUMBER) + ("op":OPERATOR) + ("lp":LPAREN)
                     + ("rp":RPAREN) + ("id":ID)

\end{verbatim}

\begin{itemize}
  \item \verb~(a3+3)*b~  = YES, \verb~ lp:(, id:a, op:+, num:3, rparen:), op:*, id:b ~
  \item \verb~)()++-33~ = YES, \verb~ rp:), lp:(, rp:), op:+, op:+, num:-33 ~
  \item \verb~(b42/3)*3~  = NO, \verb~/~ is not an accepted token.
\end{itemize}

\addcontentsline{toc}{section}{Question 5}
\section*{Question 5}
Q: Suppose the context-free grammar $G$:
\begin{Verbatim}[mathescape,commandchars=\\\{\}]
    S ::= A.S.B | B.S.A | $\epsilon$
    A ::= a | $\epsilon$
    B ::= b
\end{Verbatim}
where the starting symbol is $S$, which of the following are in the language of $G$?:\\
\\
A:
\begin{multicols}{2}
\begin{itemize}
    \item $a$ - No, a must be followed by b
    \item $b$ - Yes
    \item $ab$ - Yes
    \item $ba$ - Yes
    \item $bb$ - Yes (B . (B . S . A) . A)
    \item $baa$ - No, a must be followed by b
\end{itemize}
\end{multicols}

\addcontentsline{toc}{section}{Question 6}
\section*{Question 6}
Q: Suppose the context-free grammar $S \; := \; a.S.a \;|\; b.S.b \;|\; \epsilon$ \\
Describe the language generated by this grammar. \\
\\
A: All even-length palindromes over the alphabet \{a,b\}.

\addcontentsline{toc}{section}{Question 7 (Optional)}
\section*{Question 7 (Optional)}
Q: Prove by induction on $r$ that the property $L(der \; c \; r) = Der \; c \; (L(r))$ holds. \\
\\
A:
$Der(c, A) \equiv \{s \; | \; c::s \in A\}$, i.e. the derivate of language A w.r.t character c
is the language of all strings in A beginning with c, with the c `chopped off' the front. \\

Basis:
\begin{itemize}
    \item $R=0$: $der(c, R) = 0$ and $L(0) = \{\}$. $Der(c, \{\})=\{\}$ so holds.
    \item $R=1$: $der(c, R) = 0$, $L(0)$ as above, so holds.
    \item $R=c_1$: $der(c, c_1) = 1 \; if c=c_1 \; else \; 0\}$.\\
    $L(1) = \{[]\}$ and $L(0) = \{\}$. \\
    $Der(c, \{c\})=\{[]\}$ and $Der(c, \{d\})=\{\}$ so holds.
\end{itemize}

Inductive Hypothesis: \\

Assume some expressions $r_1$, $r_2$ such that $L(der(c, r_i)) \equiv Der(c, L(r_i))$ holds for $i = 1, 2$.
Show true for remaining regex cases. \\
\\
Inductive Proof:
\begin{itemize}
    \item $R=r_1+r_2$ \\
    The language accepted by $r_1+r_2$ is $L(r_1) \cup L(r_2)$.\\
    The language accepted by $der(c, r_1+r_2)$ is $L(der(c,r_1)) \cup L(der(c,r_2))$.\\
    Since by IH $L(der(c, r_i)) \equiv Der(c, L(r_i))$ holds, \\
    $Der(c, L(r_1+r_2)) = L(der(c,r_1)) \cup L(der(c,r_2)) = L(der(c, r_1+r_2))$ also holds.

    \item $R=r_1\;.\;r_2$ \\
    The language accepted by $r_1.r_2$ is $L(r_1) @ L(r_2)$.\\
    The language accepted by $der(c, r_1+r_2)$ is \\
    $L(((der(c, r_1)).r_2) + der(c, r_2))$ if $\epsilon \in L(r_1)$,
    else $L((der(c, r_1)).r_2)$.\\

    The second case is equivalent to $L(der(c,r_1)) @ L(r_2)$
    and the first case is equivalent to $(L(der(c,r_1)) @ L(r_2)) \cup L(der(c,r_2))$

    Since by IH $L(der(c, r_i)) \equiv Der(c, L(r_i))$ holds, \\
    when $\epsilon \in L(r_1)$: \\
    $Der(c, L(r_1.r_2)) = (L(der(c,r_1)) @ L(der(c,r_2))) \cup L(der(c,r_2)) = L(der(c, r_1+r_2))$ also holds,
    as does $Der(c, L(r_1.r_2)) = L(der(c,r_1)) @ L(der(c,r_2)) = L(der(c, r_1+r_2))$ otherwise.


    \item $R=r_1*$\\
    $D = der(c, R) = der(c, r_1*)\;.\;r*$\\
    $L(D) = L(der(c, r_1)) \cup L(R)$ \\
    $Der(c, r*) = Der(c, L(r)) \cup L(r*) = L(D)$\\
    Holds, by all previous cases as $r_i$ can only be of the forms:\\
     $r:=0|1|c|r_1+r_2|r_1.r_2|r^*$.

\end{itemize}

\end{document}
