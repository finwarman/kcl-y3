% !TeX root = ./04_answers.tex

\documentclass[english]{scrartcl}
\usepackage[T1]{fontenc}
\usepackage[utf8]{inputenc}
\usepackage{multicol}
\usepackage{paralist}
\usepackage{hyperref}
\usepackage{listings}
\usepackage{fvextra} % loads also fancyvrb
\usepackage{pgfgantt}
\usepackage{amsmath}
\usepackage{xcolor}
\usepackage{dirtytalk}
\usepackage{xparse}
\usepackage{xpatch}
\usepackage[bottom=1.2in,top=1.2in]{geometry}
\usepackage{tikz}
\usetikzlibrary{arrows,automata,positioning}
\hypersetup{
    colorlinks,
    citecolor=black,
    filecolor=black,
    linkcolor=black,
    urlcolor=black
}
\lstset{
  basicstyle=\ttfamily,
  mathescape
}

\DeclareMathVersion{ttmath}
\DeclareSymbolFont{latinletters}{OT1}{\ttdefault}{m}{n}
%\SetSymbolFont{latinletters}{ttmath}{OT1}{\ttdefault}{m}{n}
\SetSymbolFont{letters}{ttmath}{OML}{ccm}{m}{it}
\SetSymbolFont{symbols}{ttmath}{OMS}{ccsy}{m}{n}
\SetSymbolFont{largesymbols}{ttmath}{OMX}{ccex}{m}{n}

% https://tex.stackexchange.com/questions/150965/insert-symbols-inside-verbatim-mode-latex/156419

\newcommand{\changeletters}{%
  \count255=`A
  \advance\count255 -1
  \loop\ifnum\count255<`Z
    \advance\count255 1
    \mathcode\count255=\numexpr\number\symlatinletters*256+\count255\relax
  \repeat
  \count255=`a
  \advance\count255 -1
  \loop\ifnum\count255<`z
    \advance\count255 1
    \mathcode\count255=\numexpr\number\symlatinletters*256+\count255\relax
  \repeat
  \count255=`0
  \advance\count255 -1
  \loop\ifnum\count255<`9
    \advance\count255 1
    \mathcode\count255=\numexpr\number\symlatinletters*256+\count255\relax
  \repeat
}

\xapptocmd{\ttfamily}{\mathversion{ttmath}\changeletters}{}{}

\setkomafont{disposition}{\normalfont\bfseries}
\definecolor{light-gray}{gray}{0.95}

\begin{document}

\NewDocumentCommand{\codeword}{v}{%
\colorbox{light-gray}{\texttt{\textcolor{black}{#1}}}%
}

% == START: TITLE == %

\subtitle{6CCS3CFL - Compilers \& Formal Languages}
\title{Homework 4}
\author{Finley Warman}
\date{\today}

\maketitle

% == END: TITLE == %
% == START: CONTENTS == %

\tableofcontents
\par\noindent\rule{\textwidth}{0.4pt}

% == END: CONTENTS == %

\newpage

% MAIN CONTENT:

\addcontentsline{toc}{section}{Question 1}
\section*{Question 1}
Q: Give \textit{all} the values and indicate which one is POSIX for how these expressions can recognise strings.
\\
\\
A: \verb~ (ab + a) . (1 + b) ~ matching \verb~ab~.
\begin{itemize}
  \item \verb~ Sequ( Left(Sequ(Chr(a),Chr(b))), Left(Empty) ) ~ (\textit{POSIX})
  \item \verb~ Sequ( Right(Chr(a)), Right(Chr(b)) ) ~
\end{itemize}

A: \verb~ (aa + a)* ~ matching \verb~aaa~.
\begin{itemize}
  \item \verb~ Stars([ Left(Sequ(Chr(a), Chr(a))), Right(Chr(a)) ]) ~ (\textit{POSIX})
  \item \verb~ Stars([ Right(Chr(a)) ]) ~
  \item \verb~ Stars([ Left(Sequ(Chr(a), Chr(a))) ]) ~
\end{itemize}


\addcontentsline{toc}{section}{Question 2}
\section*{Question 2}
Q: If a regular expression r does not contain any occurence of \textbf{0}, is it possible for L(r) to be empty?
\\
\\
Assuming that an expression can only be defined in terms of a valid alphabet (i.e. there is no 'unmatchable character'), and negation is not allowed, then L(r) cannot be empty.
This is because excluding \textbf{0}, all expressions are defined recursively in terms of either \textbf{1} or the empty string, and if either of these is accepted then the language cannot be empty.
\\
If negation is allowed, then an example of an empty language would be \verb~ r . ~r ~., or the union of the language of r, and its complement. (Which is always empty)



\addcontentsline{toc}{section}{Question 3}
\section*{Question 3}
Q: Define tokens for a language with numbers, parenthesis, and operations. Can the given strings be lexed?
\\
\\
A: Token / Expression Defs:
\begin{verbatim}
  DIGIT       = RANGE("0123456789")
  START_DIGIT = RANGE("123456789")
  NUMBER      = DIGIT + (START_DIGIT . DIGIT*)

  LPAREN      = CHAR('(')
  RPAREN      = CHAR(')')

  OPERATOR    = CHAR('+') + CHAR('-') + CHAR('*')

  LOWERCASE   = RANGE("abcdefghijklmnopqrstuvwxyz")
  ID          = LOWERCASE . LOWERCASE*

  LANG        = ("num":NUMBER) + ("op":OPERATOR) + ("lp":LPAREN) + ("rp":RPAREN) + ("id":ID)
\end{verbatim}

\begin{itemize}
  \item \verb~(a+3)*b~  = YES, \verb~ lp:(, id:a, op:+, num:3, rparen:), op:*, id:b ~
  \item \verb~)()++-33~ = YES, \verb~ rp:), lp:(, rp:), op:+, op:+, op:-, num:33 ~
  \item \verb~(a/3)*3~  = NO, \verb~/~ is not an accepted token.
\end{itemize}


\addcontentsline{toc}{section}{Question 4}
\section*{Question 4}
Q: Assuming $r$ is nullable, show that \verb~ 1+r+r.r == r.r ~ holds.
\\
\begin{verbatim}
  1+r+r.r as a proper tree:
    = (((1+r)+r).r)
  since r is nullable, (1+r) == r
    = (((r)+r).r) = ((r+r).r)
  since (r+r) == r:
    = (r.r) = r.r
  therefore
    1+r+r.r == r.r


\end{verbatim}

\addcontentsline{toc}{section}{Question 5 (Deleted)}
\section*{Question 5 (Deleted)}


\addcontentsline{toc}{section}{Question 6}
\section*{Question 6}
Q: Give a regular expression to match comments of the form \verb~ /* ... */ ~
\\
\\
A:
\begin{verbatim}
  SEQ(
    SEQ(CHAR('/'), CHAR('*')),
    SEQ(
      STAR(NOT(SEQ(CHAR('*'), CHAR('/')))),
      SEQ(CHAR('*'), CHAR('/'))
    )
  )
\end{verbatim}

\addcontentsline{toc}{section}{Question 7}
\section*{Question 7}
Q: Simplify the given expression. Does simplification always preserve the meaning of a regular expression? \\
\newpage
A:
\begin{verbatim}
  Simplifying (0.(b.c))+((0.c)+1):
  Using 0.r = 0:
      (0) + ((0)+1)
    = 0 + (0 + 1)
  Using 0+r = r:
      0 + (1)
    = 0 + 1
    = 1
\end{verbatim}

The regex produced by simplification will be equivalent to its pre-simplified form, in that they will accept the same language.\\
However, the resulting expression may vary in \textit{how} it matches a string, and thus (unless steps are taken to rectify this), the returned matching value may be different.

\addcontentsline{toc}{section}{Question 8}
\section*{Question 8}
Q: What is \textit{mkeps} for the following expressions?
A:
\begin{itemize}
  \item \verb~(0.(b.c))+((0.c)+1)~ = $Right(Right(Empty))$
  \item \verb~(a+1).(1+1)~ = $Sequ(Right(Empty),Left(Empty))$
  \item \verb~a*~ = $Stars(Nil)$
\end{itemize}

\addcontentsline{toc}{section}{Question 9}
\section*{Question 9}
Q: What is the purpose of the record regular expression in the Sulzmann \& Lu Algorithm? \\
\\
A: When tokenizing an expression (e.g. splitting into its component words), the record expression is used for classifying these tokens. \\
e.g. when lexing a block of code, we can produce a resulting expression of records which label each (notable) sub-expression with the token they matched.

\addcontentsline{toc}{section}{Question 9}
\section*{Question 9}
Q: Define recursive functions $atmostempty$, $somechars$, $infinitestrings$. (Recalling $nullable$ and $zeroable$). \\
\\
\begin{verbatim}
atmostempty -
  atmostempty(0):     true
  atmostempty(1):     true
  atmostempty(c):     false
  atmostempty(r1+r2): atmostempty(r1) && atmostempty(r2)
  atmostempty(r1.r2): atmostempty(r1) || atmostempty(r2)
  atmostempty(r*):    atmostempty(r)
\end{verbatim}
\begin{verbatim}
somechars -
  somechars(0):     false
  somechars(1):     false
  somechars(c):     true
  somechars(r1+r2): somechars(r1) || somechars(r2)
  somechars(r1.r2): somechars(r1) || somechars(r2)
  somechars(r*):    somechars(r)
\end{verbatim}
\begin{verbatim}
infinitestrings -
  infinitestrings(0):     false
  infinitestrings(1):     false
  infinitestrings(c):     false
  infinitestrings(r1+r2): infinitestrings(r1) || infinitestrings(r2)
  infinitestrings(r1.r2): infinitestrings(r1) || infinitestrings(r2)
  infinitestrings(r*):    ~atmostempty(r)
\end{verbatim}

\end{document}
