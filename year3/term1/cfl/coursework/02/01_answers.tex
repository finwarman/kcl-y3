% !TeX root = ./01_answers.tex

\documentclass[english]{scrartcl}
\usepackage[T1]{fontenc}
\usepackage[utf8]{inputenc}
\usepackage{multicol}
\usepackage{paralist}
\usepackage{hyperref}
\usepackage{listings}
\usepackage{fvextra} % loads also fancyvrb
\usepackage{pgfgantt}
\usepackage{amsmath}
\usepackage{xcolor}
\usepackage{dirtytalk}
\usepackage{xparse}
\usepackage{xpatch}
\usepackage[bottom=1.2in,top=1.2in]{geometry}
\hypersetup{
    colorlinks,
    citecolor=black,
    filecolor=black,
    linkcolor=black,
    urlcolor=black
}
\lstset{
  basicstyle=\ttfamily,
  mathescape
}

\DeclareMathVersion{ttmath}
\DeclareSymbolFont{latinletters}{OT1}{\ttdefault}{m}{n}
%\SetSymbolFont{latinletters}{ttmath}{OT1}{\ttdefault}{m}{n}
\SetSymbolFont{letters}{ttmath}{OML}{ccm}{m}{it}
\SetSymbolFont{symbols}{ttmath}{OMS}{ccsy}{m}{n}
\SetSymbolFont{largesymbols}{ttmath}{OMX}{ccex}{m}{n}

% https://tex.stackexchange.com/questions/150965/insert-symbols-inside-verbatim-mode-latex/156419

\newcommand{\changeletters}{%
  \count255=`A
  \advance\count255 -1
  \loop\ifnum\count255<`Z
    \advance\count255 1
    \mathcode\count255=\numexpr\number\symlatinletters*256+\count255\relax
  \repeat
  \count255=`a
  \advance\count255 -1
  \loop\ifnum\count255<`z
    \advance\count255 1
    \mathcode\count255=\numexpr\number\symlatinletters*256+\count255\relax
  \repeat
  \count255=`0
  \advance\count255 -1
  \loop\ifnum\count255<`9
    \advance\count255 1
    \mathcode\count255=\numexpr\number\symlatinletters*256+\count255\relax
  \repeat
}

\xapptocmd{\ttfamily}{\mathversion{ttmath}\changeletters}{}{}

\setkomafont{disposition}{\normalfont\bfseries}
\definecolor{light-gray}{gray}{0.95}

\begin{document}

\NewDocumentCommand{\codeword}{v}{%
\colorbox{light-gray}{\texttt{\textcolor{black}{#1}}}%
}

% == START: TITLE == %

\subtitle{6CCS3CFL - Compilers \& Formal Languages}
\title{Coursework 1}
\author{Finley Warman}
\date{\today}

\maketitle

\begin{center}
    \textit{N.B. to run all test cases and questions, run:} `\verb|amm 02_coursework.sc all|'
\end{center}

% == END: TITLE == %
% == START: CONTENTS == %

\tableofcontents
\par\noindent\rule{\textwidth}{0.4pt}

% == END: CONTENTS == %

% MAIN CONTENT:

\newpage

\addcontentsline{toc}{section}{Question 1}
\section*{Question 1}
Q: Definitions for nullable:
\\
\begin{verbatim}
    nullable([c1,...,cn]) == false
    nullable(r+)          == nullable(r)
    nullable(r?)          == true
    nullable(r{n})        == if (n=0) true else nullable(r)
    nullable(r{..m})      == true
    nullable(r{n..})      == if (n=0) true else nullable(r)
    nullable(r{n..m})     == if (n=0) true else nullable(r)
    nullable(~r)          == not nullable(r)
\end{verbatim}


Q: Definitions for der:
\\
\begin{Verbatim}[mathescape,commandchars=\\\{\}]
    der c ([c1,...,cn]) == if c $\in$ [c1,...,cn] then 1 else 0
    der c (r+)          == (der c r) . r*
    der c (r?)          == (der c r)
    der c (r\{n\})        == if (n=0) then 0 else ((der c r) . r\{n-1\})
    der c (r\{..m\})      == if (m=0) then 0 else ((der c r) . r\{..m-1\})
    der c (r\{n..\})      == if (n=0) then ((der c r) . r\{0..\}) else ((der c r) . r\{n-1..\})
    der c (r\{n..m\})     == if (n=0 and m=0) then 0
                             elif (n=0) then ((der c r) . r\{0..m-1\})
                             else ((der c r) . r\{n-1..m-1\})
    der c (~r)          == ~(der c r)
\end{Verbatim}

(Further transformations are possible, such as \verb~r{0..} -> r*~ and \verb~r{0..m} -> r{..m}~, however these are not implemented in order to keep the recursive definitions simple.)

\mbox{}\\

Q: Test Table Results: \\
A: (This can be generated by running `\verb|amm 01_coursework.sc question3|')
\footnotesize\begin{verbatim}
 string | a?  | ~a  | a{3}  | (a?){3}  | a{..3}  | (a?){..3}  | a{3..5}  | (a?){3..5}  | a{0}  |
--------+-----+-----+-------+----------+---------+------------+----------+-------------+-------+
[]      | YES | YES | -     | YES      | YES     | YES        | -        | YES         | YES   |
a       | YES | -   | -     | YES      | YES     | YES        | -        | YES         | -     |
aa      | -   | YES | -     | YES      | YES     | YES        | -        | YES         | -     |
aaa     | -   | YES | YES   | YES      | YES     | YES        | YES      | YES         | -     |
aaaa    | -   | YES | -     | -        | -       | -          | YES      | YES         | -     | <-(extra)
aaaaa   | -   | YES | -     | -        | -       | -          | YES      | YES         | -     |
aaaaaa  | -   | YES | -     | -        | -       | -          | -        | -           | -     |
\end{verbatim}
\normalsize

Additional test cases for each rexp type can be checked by running:
\begin{center}
    \verb|amm 01_coursework.sc unitTests|
\end{center}

These tests pass, so the results produced are as I expected!

\addcontentsline{toc}{section}{Question 4}
\section*{Question 4}
Q: Definitions for nullable, der, and cfun-related functions.
\\
\\
A: I implemented CFUN after the initial CHAR implementation, and used \verb|CFUN(_CHAR(c))|, \verb|CHAR2(c)|, etc. only after implementing them.\\
\\
To run CFUN tests: `\verb|amm 01_coursework.sc question4|' \\
This adds \verb|CFUN|:
\begin{verbatim}
    case class CFUN(f: Char => Boolean) extends Rexp
    def nullable ... case CFUN(f) => false
    der der      ... case CFUN(f) => if (f(c)) ONE else ZERO
\end{verbatim}
alongside the following functions for \verb|char, range, all|:
\begin{verbatim}
    def _char(ch: Char): Char => Boolean           = { (c: Char) => {(ch == c)} }
    def _range(chars: Set[Char]) : Char => Boolean = { (c: Char) => {chars.contains(c)} }
    def _all() : Char => Boolean                   = { (c: Char) => true }
\end{verbatim}
and these specific instances of CFUN to replace the existing CHAR, RANGE, etc.:
\begin{verbatim}
    def CHAR2(c: Char)           = CFUN(_char(c))
    def RANGE2(chars: Set[Char]) = CFUN(_range(chars))
    val ALL                      = CFUN(_all())
\end{verbatim}

\mbox{}\\
Example: \verb|SEQ(CFUN(_CHAR('a')), SEQ(CFUN(_RANGE(Set('b', 'B'))), STAR(CFUN(_ALL))))|
matches: $a[bB].*$, as does \verb|CHAR2('a') o RANGE2(Set('b', 'B')) o STAR(ALL)| \\
\textit{(using custom `o' infix notation for SEQ)}

\addcontentsline{toc}{section}{Question 5}
\section*{Question 5}
Q: Email Address Regular Expressions and Derivative w.r.t. my email.
\\
\\
A: (To run: `\verb|amm 01_coursework.sc question5|') \\
Ders \verb~"finley.warman@kcl.ac.uk" ~$([-.\_0{\text -}9a{\text -}z]^+\cdot(@\cdot([-.0{\text -}9a{\text -}z]^+\cdot(.\cdot[.a{\text -}z]^{\{2..6\}}))))$:
\[
    ((([-.0{\text -}9a{\text -}z]^* \cdot (. \cdot [.a{\text -}z]^{\{2..6\}})) + [.a{\text -}z]^{\{0..4\}}) + [.a{\text -}z]^{\{0..1\}})
\]
This final derivative matches the empty string $\varepsilon$, therefore the Email Rexp matches the input string of my email address.

\addcontentsline{toc}{section}{Question 6}
\section*{Question 6}
Q: Determine whether the following match the expression
$ / \cdot * \cdot (\char`\~(ALL^* \cdot * \cdot / \cdot ALL^*)) \cdot * \cdot /$
\\
\\
A: (To run: `\verb|amm 01_coursework.sc question6|') \\
\begin{itemize}
    \item matches \verb|/**/|? - \textit{YES}
    \item matches \verb|/*foobar*/|? - \textit{YES}
    \item matches \verb|/*test*/test*/|? - \textit{NO}
    \item matches \verb|/*test/*test*/|? - \textit{YES}

\end{itemize}

\addcontentsline{toc}{section}{Question 7}
\section*{Question 7}
Q: Determine whether the following match the expressions $ r_1 = a \cdot a \cdot a$ and
$ r_2 = (a^{\{19,19\}}) \cdot (a^?)$ when in the form $({r_1}^+)^+$ and $({r_2}^+)^+$.
\\
\\
A: (To run: `\verb|amm 01_coursework.sc question7|')
\begin{itemize}
    \item $({r_1}^+)^+$ matches $5.$? - \textit{YES}
    \item $({r_1}^+)^+$ matches $6.$? - \textit{NO}
    \item $({r_1}^+)^+$ matches $7.$? - \textit{NO}
    \item $({r_2}^+)^+$ matches $5.$? - \textit{YES}
    \item $({r_2}^+)^+$ matches $6.$? - \textit{NO}
    \item $({r_2}^+)^+$ matches $7.$? - \textit{YES}
\end{itemize}


\end{document}
