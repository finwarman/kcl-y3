% !TeX root = ./02_answers.tex

\documentclass[english]{scrartcl}
\usepackage[T1]{fontenc}
\usepackage[utf8]{inputenc}
\usepackage{multicol}
\usepackage{paralist}
\usepackage{hyperref}
\usepackage{listings}
\usepackage{fvextra} % loads also fancyvrb
\usepackage{pgfgantt}
\usepackage{amsmath}
\usepackage{xcolor}
\usepackage{dirtytalk}
\usepackage{xparse}
\usepackage{xpatch}
\usepackage[bottom=1.2in,top=1.2in]{geometry}
\hypersetup{
    colorlinks,
    citecolor=black,
    filecolor=black,
    linkcolor=black,
    urlcolor=black
}
\lstset{
  basicstyle=\ttfamily,
  mathescape
}

\DeclareMathVersion{ttmath}
\DeclareSymbolFont{latinletters}{OT1}{\ttdefault}{m}{n}
%\SetSymbolFont{latinletters}{ttmath}{OT1}{\ttdefault}{m}{n}
\SetSymbolFont{letters}{ttmath}{OML}{ccm}{m}{it}
\SetSymbolFont{symbols}{ttmath}{OMS}{ccsy}{m}{n}
\SetSymbolFont{largesymbols}{ttmath}{OMX}{ccex}{m}{n}

% https://tex.stackexchange.com/questions/150965/insert-symbols-inside-verbatim-mode-latex/156419

\newcommand{\changeletters}{%
  \count255=`A
  \advance\count255 -1
  \loop\ifnum\count255<`Z
    \advance\count255 1
    \mathcode\count255=\numexpr\number\symlatinletters*256+\count255\relax
  \repeat
  \count255=`a
  \advance\count255 -1
  \loop\ifnum\count255<`z
    \advance\count255 1
    \mathcode\count255=\numexpr\number\symlatinletters*256+\count255\relax
  \repeat
  \count255=`0
  \advance\count255 -1
  \loop\ifnum\count255<`9
    \advance\count255 1
    \mathcode\count255=\numexpr\number\symlatinletters*256+\count255\relax
  \repeat
}

\xapptocmd{\ttfamily}{\mathversion{ttmath}\changeletters}{}{}

\setkomafont{disposition}{\normalfont\bfseries}
\definecolor{light-gray}{gray}{0.95}

\begin{document}

\NewDocumentCommand{\codeword}{v}{%
\colorbox{light-gray}{\texttt{\textcolor{black}{#1}}}%
}

% == START: TITLE == %

\subtitle{6CCS3CFL - Compilers \& Formal Languages}
\title{Coursework 2}
\author{Finley Warman}
\date{\today}

\maketitle

\begin{center}
    \textit{N.B. to run all test cases and questions, run:} `\verb|amm 02_coursework.sc all|'
\end{center}

% == END: TITLE == %
% == START: CONTENTS == %

\tableofcontents
\par\noindent\rule{\textwidth}{0.4pt}

% == END: CONTENTS == %

% MAIN CONTENT:

\newpage

\addcontentsline{toc}{section}{Question 1}
\section*{Question 1}
Implemented `simple' and `extended' regular expressions, including REC, in \verb~02_coursework.sc~.
Expressions added for all tokens described, including expression \verb~WHILE_LANG_REG~ representing all valid tokens
of the WHILE language.\\
(See KEYWORD, OP, ID, SEMI, NUMBER, COMMENT, STRING, LPAREN, RPAREN, LCURL, RCURL, WHITESPACE)

\addcontentsline{toc}{section}{Question 2}
\section*{Question 2}
Q: Definitions for mkeps:
\begin{verbatim}
    ...
    mkeps([c1,...,cn]) == N/A, not nullable
    mkeps(r+)          == Plus[mkeps(r)]
    mkeps(r?)          == Empty
    mkeps(r{n})        == if (n=0) NTimes[] else NTimes[mkeps(r)]
\end{verbatim}


Q: Definitions for inj:
\begin{verbatim}
    ...
    inj([c1,...,cn]) c Empty        == Char(c)
    inj(r+) c Seq(v, Stars vs)      == Plus(inj(r) c v :: vs)
    inj(r?) c v                     == Opt(inj(r) c v)
    inj(r{0}) c v                   == Empty
    inj(r{n}) c Seq(v, NTimes vs)   == NTimes(inj(r) c v :: vs)
\end{verbatim}

To test $a^{\{3\}}$ and $(a+\textbf{1})^{\{3\}}$, call \verb|amm 02_coursework.sc lex_extended|.\\
\\
Tokens for expressions from Q1:
\begin{verbatim}
WHILE_LANG_REG = (
  ("kwd" : KEYWORD) +
  ("op"  : OP)      +
  ("id"  : ID)      +
  ("semi": SEMI)    +
  ("num" : NUMBER)  +
  ("comm": COMMENT) +
  ("str" : STRING)  +
  ("brkt": (LPAREN + RPAREN)) +
  ("crly": (LCURL  + RCURL )) +
  ("wspc": WHITESPACE)
)*
\end{verbatim}

The token sequence for \verb|read n;| (\verb|amm 02_coursework.sc while_lang_lexing_tests|):
\begin{verbatim}
    kwd:     "read"
    wspc:    " "
    id:      "n"
    semi:    ";"
\end{verbatim}

\addcontentsline{toc}{section}{Question 3}
\section*{Question 3}
Resulting tokens for each program, filtering whitespace \verb~wspc~:\\
(\verb|amm 02_coursework.sc program_lexing_tests|)
\\
\begin{verbatim}
    ========== collatz.while ==========

    kwd:     "write"
    str:     "\"Input a number \""
    semi:    ";"
    kwd:     "read"
    id:      "n"
    semi:    ";"
    kwd:     "while"
    id:      "n"
    op:      ">"
    num:     "1"
    kwd:     "do"
    crly:    "{"
    kwd:     "if"
    id:      "n"
    op:      "%"
    num:     "2"
    op:      "=="
    num:     "0"
    kwd:     "then"
    id:      "n"
    op:      ":="
    id:      "n"
    op:      "/"
    num:     "2"
    kwd:     "else"
    id:      "n"
    op:      ":="
    num:     "3"
    op:      "*"
    id:      "n"
    op:      "+"
    num:     "1"
    semi:    ";"
    crly:    "}"
    semi:    ";"
    kwd:     "write"
    str:     "\"Yes\""
    semi:    ";"

    ===================================
\end{verbatim}
\newpage
\begin{verbatim}
    ========== factors.while ==========

    comm:    "// Find all factors of a given input number\n"
    comm:    "// by J.R. Cordy August 2005\n"
    kwd:     "write"
    str:     "\"Input n please\""
    semi:    ";"
    kwd:     "read"
    id:      "n"
    semi:    ";"
    kwd:     "write"
    str:     "\"The factors of n are\""
    semi:    ";"
    id:      "f"
    op:      ":="
    num:     "2"
    semi:    ";"
    kwd:     "while"
    id:      "n"
    op:      "!="
    num:     "1"
    kwd:     "do"
    crly:    "{"
    kwd:     "while"
    brkt:    "("
    id:      "n"
    op:      "/"
    id:      "f"
    brkt:    ")"
    op:      "*"
    id:      "f"
    op:      "=="
    id:      "n"
    kwd:     "do"
    crly:    "{"
    kwd:     "write"
    id:      "f"
    semi:    ";"
    id:      "n"
    op:      ":="
    id:      "n"
    op:      "/"
    id:      "f"
    crly:    "}"
    semi:    ";"
    id:      "f"
    op:      ":="
    id:      "f"
    op:      "+"
    num:     "1"
    crly:    "}"

    ===================================
\end{verbatim}
\newpage
\begin{verbatim}
    ==========  loops.while  ==========

    id:      "start"
    op:      ":="
    num:     "1000"
    semi:    ";"
    id:      "x"
    op:      ":="
    id:      "start"
    semi:    ";"
    id:      "y"
    op:      ":="
    id:      "start"
    semi:    ";"
    id:      "z"
    op:      ":="
    id:      "start"
    semi:    ";"
    kwd:     "while"
    num:     "0"
    op:      "<"
    id:      "x"
    kwd:     "do"
    crly:    "{"
    kwd:     "while"
    num:     "0"
    op:      "<"
    id:      "y"
    kwd:     "do"
    crly:    "{"
    kwd:     "while"
    num:     "0"
    op:      "<"
    id:      "z"
    kwd:     "do"
    crly:    "{"
    id:      "z"
    op:      ":="
    id:      "z"
    op:      "-"
    num:     "1"
    crly:    "}"
    semi:    ";"
    id:      "z"
    op:      ":="
    id:      "start"
    semi:    ";"
    id:      "y"
    op:      ":="
    id:      "y"
    op:      "-"
    num:     "1"
    crly:    "}"
    semi:    ";"
    id:      "y"
    op:      ":="
    id:      "start"
    semi:    ";"
    id:      "x"
    op:      ":="
    id:      "x"
    op:      "-"
    num:     "1"
    crly:    "}"

    ===================================
\end{verbatim}
\newpage
\begin{verbatim}
    ========= collatz2.while ==========

    comm:    "// Collatz series\n"
    comm:    "//\n"
    comm:    "// needs writing of strings and numbers; comments\n"
    id:      "bnd"
    op:      ":="
    num:     "1"
    semi:    ";"
    kwd:     "while"
    id:      "bnd"
    op:      "<"
    num:     "101"
    kwd:     "do"
    crly:    "{"
    kwd:     "write"
    id:      "bnd"
    semi:    ";"
    kwd:     "write"
    str:     "\": \""
    semi:    ";"
    id:      "n"
    op:      ":="
    id:      "bnd"
    semi:    ";"
    id:      "cnt"
    op:      ":="
    num:     "0"
    semi:    ";"
    kwd:     "while"
    id:      "n"
    op:      ">"
    num:     "1"
    kwd:     "do"
    crly:    "{"
    kwd:     "write"
    id:      "n"
    semi:    ";"
    kwd:     "write"
    str:     "\",\""
    semi:    ";"
    kwd:     "if"
    id:      "n"
    op:      "%"
    num:     "2"
    op:      "=="
    num:     "0"
    kwd:     "then"
    id:      "n"
    op:      ":="
    id:      "n"
    op:      "/"
    num:     "2"
    kwd:     "else"
    id:      "n"
    op:      ":="
    num:     "3"
    op:      "*"
    id:      "n"
    op:      "+"
    num:     "1"
    semi:    ";"
    id:      "cnt"
    op:      ":="
    id:      "cnt"
    op:      "+"
    num:     "1"
    crly:    "}"
    semi:    ";"
    kwd:     "write"
    str:     "\" => \""
    semi:    ";"
    kwd:     "write"
    id:      "cnt"
    semi:    ";"
    kwd:     "write"
    str:     "\"\\n\""
    semi:    ";"
    id:      "bnd"
    op:      ":="
    id:      "bnd"
    op:      "+"
    num:     "1"
    crly:    "}"

    ===================================
\end{verbatim}

\end{document}
